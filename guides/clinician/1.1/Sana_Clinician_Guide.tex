\documentclass[a4paper,10pt]{article}
\usepackage[utf8x]{inputenc}

%opening
\title{Sana Clinician's Guide}
\author{}

\begin{document}

\maketitle

\begin{abstract}

\end{abstract}

\section{Client User Guide: Clinicians}
Targeted at a clinician who does not need to manage the information system. The guide will walk through the transfer of a paper-based medical questionaire into a format usable on a Sana mobile client. 

\subsection{Types of Input Available}
Element type, description, image, Cut+paste XML sample  (HIGHLIGHT AREAS TO BE CHANGED)
------------------------------------------------------
Text data,  
Single choice,  
Multi choice,  
Images,

Note: See the Advanced Content Creation Guide for more details and assitance. 

\subsection{Medical Decision Trees}
Blurb about what they are, and how to use them.

- If, then logic
If/then logic is a way of choosing to display or not display certain questions based on previous questions. 
{IMAGE: History of smkoking? T/F} BRANCH T-> {IMAGE: Take an image of your mouth} 
				  BRANCH F-> {IMAGE: History of drinking?}
XML Code sample

- Comparison operators
Comparison operators are a more powerful way of using if/then logic. Rather than only checking for a simple equality, we can compare an input value to another pre-set value. 

{IMAGE: What is your age?} ->  {IMAGE: WHat is your BP?} IF > 40  AND BP > 120 {IMAGE: You are at risk for Blah}
									       {IMAGE: Your risk is minimal}

\subsection{Translating Document}
Based on the information above, a significant number of decision trees can be tranlated into XML docuemnts. 

Example: Side by side, color coded for each question
{IMAGE: Decisions tree with nodes}  ->  {IMAGE: XML code}


\subsection{Get Content to Phone}
Upload it to a webserver 
{IMAGE: Webpage that has 3 XML docs}

Note: You may want to talk with an IT expert if unsure how/where you can upload a document to the web.

Browse to the location on your phone.
{IMAGE: Same webpage on a phone}

Download the form to your phone.
{IMAGE: Select a form --highlight it}

Move the form to directory
{IMAGE: Mobile phone with the file browser open to a given directory}

Load form into the Sana App
{IMAGE: Splash with menu and settings visible}
{IMAGE: First settings page, higlight Sana Resources}
{IMAGE: Second settings page, highlight}

\subsection{Setup Server with Sana Components}
Consult with someone who can setup the server for you, or consult the System Administrator Guide if settign up yourself. 

\subsection{Collect Data in the Field}
Clinical health workers in the field will now collect field data using your cusotm procedure, and the logic in the decision trees. 

Note: If you're unsure of how to install or use the mobile client, see the Client User Guide.

\subsection{Manage Patient Encounters}
Login to your server using your account.

Login to OpenMRS and navigate to the Sana Queue.
{IMAGE: Screenshot of Sana Queue}

You should see data collected using the forms you just built.
{IMAGE: Show one upladed of the same type that we did with the ``Getting content to phone'' section}

This will launch a screen for you to review the data and respond to your healthcare worker.
{IMAGE: Sana mediaviewer page}

Review patient data to determine diagnois or request additional ingformation from the worker. 
{IMAGE: SEnd with diagnois information} {IMAGE: HIGHLIGHT a ``take a picture of patient's foot'' message}

Note: Notification will be returned to the phone. If you need more information about the notifivcation process, speak to your sys admin. 

\clearpage{}



\end{document}
