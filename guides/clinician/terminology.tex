\documentclass[a4paper,10pt]{article}
\usepackage[utf8]{inputenc}

%opening
\title{Sana Terminology v1.x}
\author{Sana}


\begin{document}
\section{Terminology}
Terminology used in the Sana platform.  

\subsection{General}
The following terms define the terms which are used throughout the Sana 
documentation to describe the process of data collection.
\begin{description}
 \item[Subject] Who or what data is collected about 
 \item[Worker] An entity that collects data about a subject
 \item[Instruction] Provides context for a single instance of data collection. 
 \item[Procedure] A set of one or more instructions to be executed.
 \item[Observation] The data collected as a result of executing an instruction.
 \item[Encounter] A set of one or more observations which are collected during
 the execution of a Procedure.
 \item[Concept] A functional unit of meaning. Provide context to data that is
 collected.
 \item[Ontology] The collective set of concepts which provide meaning within
 the domain in which data is collected.
 \item[UUID] A universally unique identifier.
\end{description}

\subsection{Medical}
The following terms are extensions of the more general terminology and 
\begin{description}
 \item[Patient] A subject about whom medical data is collected.
 \item[Clinical Health Worker, CHW] The entity which executes the intructions.
 \item[SavedProcedure] A synonym for \textbf{Encounter}.
\end{description}

\subsection{Procedure Mark Up} 
The following terms are the XML node tags and atttributes used when implementing
a \textbf{Procedure} as an XML formatted document. Attributes which are optional
are noted as such. Please see the \textbf{Clinician Guide} for full details of 
use.
\begin{description}
 \item [Procedure] Marks the beginning of a \textbf{Procedure}.
  \begin{description}
   \item[title] A descriptive name.
   \item[author] Who wrote the document.
   \item[uuid(Optional)] The id of the document.
   \item[version(Optional)] Distinguishes different versions of a protocol.
  \end{description}  
 \item [Page] Marks a set of one or more instructions which are viewed 
 collectively on a single page.
 \item[ShowIf] The entry point for branching logic which must logically 
 evaluate to true for a Page to be displayed.
 \item[Criteria] An expression which will yield a logical truth value when
 evaluated.
  \begin{description}
   \item[id] The id of the node to be used for comparison.
   \item[type] The type of comparison operator.
   \item[value] A value to compare against.
  \end{description}  
 \item[and] Logical conjunction between two or more \textbf{Criteria}.
 \item[or] Logical disjunction between two or more \textbf{Criteria}.
 \item[not] Logical negation of the result of a simple or compound 
 \textbf{Criteria}. 
 \item [Element] A single \textbf{Instruction} node.
  \begin{description}
   \item[id] A unique identifier within the \textbf{Procedure}. The value is an
   arbitrary string but should limited to alphanumeric characters with no
   whitespace. 
   \item[type] A classification of the visual element to use.
   \item[concept] A term which provides context for the instruction.
   \item[question] The id of the document.
   \item[answer] A default value for the collected data.
   \item[choices] A  comma separated list of one or more options to choose from.
   Required for SELECT, RADIO, MULTI_SELECT.
  \end{description}  
\end{description}
\end{document}
